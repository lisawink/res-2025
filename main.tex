\documentclass[12pt]{article}
\usepackage[utf8]{inputenc}
\usepackage[english]{babel}
\usepackage{setspace}
\usepackage{geometry}
\usepackage{amsmath}
\usepackage{graphicx}
\usepackage{hyperref}
\usepackage{csquotes}
\usepackage{caption}
\usepackage{subcaption}
\usepackage{booktabs}
\usepackage[utf8]{inputenc}
\usepackage{csquotes}
\usepackage{amsmath}
\usepackage{graphicx}
\usepackage{float}

% enhanced citation package 
\usepackage[style=apa, backend=biber]{biblatex}

% Pfade zu deinen Bibliografie-Dateien angeben
\addbibresource{GundD.bib}

%Zitat in Klammern: \parencite{Doe2015}.

% hyperrefs on, with nicer colors
\usepackage{color}
\usepackage{xcolor}
\usepackage[]{hyperref}
\definecolor{darkblue}{rgb}{0,0,.5}
\hypersetup{colorlinks=true, breaklinks=true, linkcolor=darkblue, menucolor=darkblue, urlcolor=darkblue, citecolor=darkblue}

% Set line spacing to 1.5
\setstretch{1.5}

% Set page margins
\geometry{a4paper, margin=1.8cm}

%----------------------
\begin{document}

% Title Page
\begin{titlepage}
    \begin{center}
    \includegraphics[width=0.3\textwidth]{symbol_nmbu.jpeg}\par\vspace{1cm}
        \vspace*{1cm}

        {\LARGE\bfseries Navigating Change: \par
        The Role of Feminist Actors and NGOs in Advancing Gender Equality in Post-Revolution Tunisia \par}

        \vspace{1cm}
        
        {\large A Critical Examination of Achievements, Challenges, and the Impact of Recent Political Developments from a Feminist and Intersectional Perspective \par}

        \vspace{2cm}

        {\large Norwegian University of Life Sciences (NMBU) \par}
        \vspace{0.5cm}
        {\large Assignment for EDS370: Gender and Development \par}

        \vfill

        {\large Annette Handschuh \par}
        \vspace{0.5cm}
        {\large Student number: 128247 \par}
        \vspace{0.5cm}
        {\large Teacher: Professor Lyla Mehta \par}

        \vfill

        {\large Assignment handed in on the 2nd of September 2024 \par}

        \vspace{2cm}
    \end{center}
\end{titlepage}


%--------------------------------------

\tableofcontents

\thispagestyle{empty}

\clearpage % Seitenumbruch
\pagenumbering{arabic} % Arabische Seitennummerierung starten
\setcounter{page}{1} % Starten der Nummerierung bei 1

\section{Introduction}

The Tunisian Revolution of 2011, a pivotal event that ignited the Arab Spring, set the stage for a new era of political and social reform in Tunisia \parencite{arfaoui_violence_2016}. ....

This essay seeks to address a critical inquiry intersecting gender and development: How have feminist actors and feminist NGOs influenced the promotion of gender equality in Tunisia since the 2011 revolution, and what are the implications of recent political developments on these efforts from a feminist and intersectional perspective? 

By exploring this question, the essay analyzes the integral role played by feminist actors and NGOs in the constitutional process from 2011 to 2014, highlighting both successes and challenges. It examines their ongoing efforts to promote gender rights, with a particular focus on addressing gender-based violence, and explores the instruments they use to achieve their goals. Additionally, the essay assesses the impact of recent democratic backlashes on the work of these feminist organizations and examines how they navigate and resist these challenges. Throughout the analysis, a comprehensive background is provided to offer a deeper understanding of the context and complexities involved. By examining these aspects, we can gain a deeper understanding of the obstacles faced, the critical role of feminist NGOs, and identify potential pathways for sustaining and advancing gender equality in Tunisia amidst evolving political dynamics.


\begin{figure}[H]
  \centering
  \includegraphics[width=0.8\textwidth, trim=0cm 11cm 0cm 0cm, clip]{frau_protest_tunesien.pdf}
  \caption{Feminist in action x3}
\end{figure}

\section{Feminist Contributions to the Constitutional Change in 2014}

In January 2014, Tunisia ratified what is widely regarded as one of the most progressive constitutions among the "Arab Spring" countries, marking a significant milestone in its transition from dictatorship to democracy \parencite{tamaru_beyond_2018}. ...

\subsection{Feminism in Tunisia}
In 1956, President Habib Bourguiba introduced ....





\section{Summary}
In summary, feminist actors and feminist NGOs have profoundly influenced the promotion of gender equality in Tunisia, primarily through their impactful mobilization and strategic alliances. Their efforts to build diverse coalitions and employ a variety of advocacy tools have significantly advanced gender equality, demonstrating the power of coordinated action and innovative strategies in achieving substantial social change.
Their ability to navigate power relations and address intersecting social factors has been and will be critical in shaping legal and policy reforms. As Tunisia faces ongoing political challenges, the adaptability and sustained efforts of these organizations will be essential for maintaining progress and ensuring that gender equality remains a priority.


\newpage

\printbibliography[title=References, heading=bibintoc]

\end{document}