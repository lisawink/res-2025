\documentclass[12pt]{article}
\usepackage[utf8]{inputenc}
\usepackage{natbib}
\usepackage{url}
\usepackage{graphicx}
\title{The Majestic \textit{Littorina littorea}}
\author{Riley Luck}
\date{\today}
\begin{document}
\maketitle

\begin{figure}[!h]
    \centering
    \includegraphics[width=0.5\linewidth]{snail.png}
    \caption{Behold the mighty \textit{L. littorea}}
    \label{fig:placeholder}
\end{figure}

The humble common periwinkle, \textit{Littorina littorea}, may seem at first glance, a mere trifle. This is not so. \textit{L. littorea}, when acting as a hive mind, indeed may have the power to rival the Gods themselves. As explained by Eric Weniger, the author of the paper of wide renown \citep{eric}, "Perwinkles, when retracting into their well constructed houses, against the very common belief of them being asleep, [actually] listen to the latest Usher mixtapes on Spotify, without a premium membership. The large abundance of non-paying periwinkles is one of the greatest threats to the Scandinavian economy." As we can, this humble sea snail has the ability to influence geopolitical tensions far beyond the scale previously assumed, before the monumental writing of this paper. The theories contained within the present paper shatter paradigms of common scientists that have been held for generations. For example, \cite{saier_age-dependent_2000}, erroneously believed they were researching a simple population of gastropods rather than a thriving, intelligent society of God-like beings.

\begin{table}[]
    \centering
    \begin{tabular}{c|c}
        Power level & 9000 + \\
        Charisma & 22.5
    \end{tabular}
    \caption{Median characteristics of \textit{L. littorea} specimens from an Eastern North American population.}
    \label{tab:placeholder}
\end{table}

\bibliographystyle{plainnat}
\bibliography{bibliography}

\end{document}
